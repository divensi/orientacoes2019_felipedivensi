\chapter{Introdução}\label{ch:intro}

A Frimesa Cooperativa Central atua no setor alimentício, que em 2017 completou 40 anos de mercado. De acordo com um relatório público de 2017, a Frimesa está entre as 300 maiores empresa do Brasil, tendo em seu quadro, aproximadamente 7 mil colaboradores diretos e faturando aproximadamente R\$ 2.83 bilhões. Possui como meta para 2018 3.2 bilhões de faturamento e com projeção estratégica de crescimento da empresa para R\$ 5.0 bilhões para 2022 \cite{relatorio2017}.

% Estamos terminando 2019 e você falou sobre 2018

A Unidade de Medianeira-PR é a sede da empresa, onde os departamentos administrativos responsáveis por toda a empresa estão localizados, como o departamento de Tecnologia da Informação \sigla{Ti}{Tecnologia da Informação}. O TI, de acordo com o relatório % QUE RELATÓRIO?, possui como objetivo ''Manter atualizada a Tecnologia da Informação da Frimesa.``.

O departamento de TI é subdividido em três setores: Suporte, Infraestrutura e Desenvolvimento de Sistemas. O último sendo o setor onde o estágio foi realizado.

\section{Atividades previstas}

As atividades previstas no plano de estágio incluem a criação e manutenção de relatorios, telas, criação de procedimentos e rotinas no Oracle E-Business Suite, utilizando PL/SQL. Programação nas linguagens Java, Shell Script, PHP e HTML, entre outras atividades a critério do superior imediato.

\section{Identificação}

Nome do acadêmico: Felipe Divensi Eckl.

Nome do orientador: Everton Coimbra de Araújo

Instituição de ensino: Universidade Tecnológica Federal do Paraná - UTFPR

Endereço da Instituição: 
Avenida Brasil, 4232 CEP 85884-000 - Caixa Postal 271.
Telefone Geral +55 (45) 3240-8000 - Fax : +55 (45) 3240-8101.
Medianeira - PR – Brasil.

Nome do curso: \UTFPRprogramadata.

Empresa: Frimesa Cooperativa Central.

Ramo de atividade da empresa: Análise e Desenvolvimento de Sistemas.

Endereço da empresa: 
Rua Bahia, 159 CEP 85884-000
Telefone +55 (45) 3264-8000 - Fax : +55 (45) 3264-8028
Medianeira - PR – Brasil

\section{Organização do documento}

Esse documento será organizado da seguinte forma. O Capítulo \ref{cap:metod} apresenta os materiais e métodos que foram utilizados no estágio. O Capítulo \ref{cap:result} irá falar sobre os projetos executados, como eles foram executados e quais os resultados obtidos. O Capítulo \ref{cap:concl} irá incluir algumas conclusões fazendo uma dissetação sobre a experiência no estágio.

