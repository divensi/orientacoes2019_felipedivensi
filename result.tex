\chapter{RESULTADOS E DISCUSSÕES} \label{cap:result}

Neste capítulo serão mostrados alguns dos projetos trabalhados e os resultados obtidos com o estágio.

\section{Projetos Trabalhados}

\subsection{Aplicação de Quiz Responsiva}

Um dos projetos desenvolvidos foi uma aplicação Web Responsiva para um Quiz da empresa (Figura \ref{fig:quiz-1}). 

% FALE SOBRE O APP. Para que serve? Quem usa? Faça isso para todos relacionados neste capítulo.


Após receber os requerimentos do projeto, foram decididas as tecnologias que seriam utilizadas na aplicação. Foi utilizado a linguagem de programação PHP com o framework CakePHP versão 3 em conjunto com o banco de dados MySQL. Também foi decidido que a página na qual os usuários que fossem interagir seria uma única página, e seu conteúdo seria atualizado utilizando JavaScript para se comunicar com uma WebAPI.

\begin{figure}[h!]
  \centering
  \includegraphics[width=0.7\textwidth]{Imagens/screenshots/quiz-1.png}
  \caption[Aplicação de Quiz Responsiva]{Aplicação de Quiz Responsiva}
  \label{fig:quiz-1}
  \fonte{(Autoria Própria)}
\end{figure}

\subsection{Portal da Intranet}

O projeto de criação do Portal da Intranet responsivo foi desenvolvido para substituir as funcionalidades de um portal anterior que fazia o mesmo papel. Este que é a página inicial da intranet da empresa com links a diversos sistemas, funcionalidades e manuais que são utilizados por funcionários de diversas áreas.

Também foi criada uma área administrativa para que os usuários que possuam permissões administrativas possam cadastrar os links e imagens que aparecem no portal. 

Este portal foi desenvolvido utilizando o \textit{framework} CakePHP versão 3, em conjunto com o banco de dados MySQL, também foi utilizado JavaScript para fazer animações e partes interativas das páginas, como um carrossel de imagens e o carregamento das notícias na página principal.

% apresente a figura


\begin{figure}[h!]
  \centering
  \includegraphics[width=0.7\textwidth]{Imagens/screenshots/portal-2.png}
  \caption[Portal da Intranet]{Portal da Intranet}
  \label{fig:portal-1}
  \fonte{(Autoria Própria)}
\end{figure}


\subsection{Portal do Representante Responsivo}

O projeto de criação do Portal de Representantes responsivo, Figura \ref{fig:portalrep-1}, foi criado para centralizar todas as informações e links para que os representantes da empresa possam utilizar enquanto estiverem trabalhando remotamente. Para isto, um dos principais requisitos da página é que ela fosse responsiva, e funcionasse em qualquer dispositivo Android e iOS atualizado. 

Esta página também teve uma página administrativa criada para que as informações deste fossem cadastradas. Este portal também foi desenvolvido utilizando o framework CakePHP versão 3 e o banco de dados MySQL, utilizando Javascript para interatividade na página inicial.

% está usando muito TAMBÉN


\begin{figure}[h!]
  \centering
  \includegraphics[width=0.7\textwidth]{Imagens/screenshots/portal-2.png}
  \caption[Portal do Representante]{Portal do Representante}
  \label{fig:portalrep-1}
  \fonte{(Autoria Própria)}
\end{figure}



\subsection{Manutenção Sistema de Chamados}

Parte das tarefas do estágio foi dar manutenção ao Sistema de Gestão de Chamados do setor de Tecnologia da Informação. Este sistema (também desenvolvido e mantido por outros funcionários) é responsável pela aplicação da metodologia ITIL da empresa. Neste foram desenvolvidas novas telas, atendendo a requisitos para melhoria do mesmo.

% E ONDE ESTÃO AS FIGURAS DO QUE FEZ? Precisa referenciar, indicar elas no texto. E também é preciso comentar sobre elas, o que são e representam.

\begin{figure}[h!]
  \centering
  \includegraphics[width=1\textwidth]{Imagens/screenshots/sgc-1.png}
  \caption[Sistema de Gestão de Chamados]{Sistema de Gestão de Chamados}
  \label{fig:sgc-1}
  \fonte{(Autoria Própria)}
\end{figure}
