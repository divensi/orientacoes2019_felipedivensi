\makeatletter
\def\input@path{{texmf/tex/latex/abntex/}}
\makeatother
\documentclass[oneside]{normas-utf-tex} %oneside = para dissertacoes com numero de paginas menor que 100 (apenas frente da folha) 

% force A4 paper format
%\special{papersize=210mm,297mm}

\usepackage[brazil]{babel} % pacote portugues brasileiro
\usepackage[utf8]{inputenc} % pacote para acentuacao direta
\usepackage{amsmath,amsfonts,amssymb} % pacote matematico
\usepackage{graphicx} % pacote grafico
%\usepackage{times} % fonte times
\usepackage[final]{pdfpages} % adicao da ata
\usepackage{hyperref} % gera hiperlinks para o sumario, links, referencias -- deve vir antes do 'abntcite'
\usepackage{url}
\usepackage{tabularx}
\usepackage{soulutf8}
\usepackage{pgfgantt}
\usepackage{rotating}
\usepackage[alf,abnt-emphasize=bf,bibjustif,recuo=0cm, abnt-etal-cite=2, abnt-etal-list=99]{abntcite} %configuracao correta das referencias bibliograficas.
\usepackage{subfigure}
\usepackage{xtab}
%\usepackage{nomencl}
\usepackage[portuguese,algoruled,longend,linesnumbered]{algorithm2e} %Para linhas numeradas e réguas de identação
\usepackage{algorithm2e}
\usepackage{listings}

% Parada de desenho
\usepackage{tikz}
% Viadagem de seta
\usetikzlibrary{arrows.meta,positioning}

% Definindo novas cores
\definecolor{verde}{rgb}{0.25,0.5,0.35}
\definecolor{jpurple}{rgb}{0.5,0,0.35}
% Configurando layout para mostrar codigos Java
%\lstset{
%  language=Java,
%  basicstyle=\ttfamily\small,
%  keywordstyle=\color{jpurple}\bfseries,
%  stringstyle=\color{red},
%  commentstyle=\color{verde},
%  morecomment=[s][\color{blue}]{/**}{*/},
%  extendedchars=true,
%  showspaces=false,
%  showstringspaces=false,
%  numbers=left,
%  numberstyle=\tiny,
%  breaklines=true,
%  backgroundcolor=\color{cyan!10},
%  breakautoindent=true,
%  captionpos=b,
%  xleftmargin=0pt,
%  tabsize=4
%}
\renewcommand{\lstlistingname}{Código}
\renewcommand\lstlistlistingname{Lista de Códigos}


% ---------- Preambulo ----------
\instituicao{Universidade Tecnol\'ogica Federal do Paran\'a} % nome da instituicao (ABNTinstituicaodata)
\departamento{Departamento Acad\^{e}mico de Computa\c{c}\~ao} % nome do programa (UTFPRdepartamentodata)
\programa{Curso de Ci\^{e}ncia da Computa\c{c}\~ao} % área ou curso (UTFPRprogramadata)

\documento{Relatório de Estágio}
\nivel{Graduação}                %  TADS e CC = Gradução; ETEC = Graduação Técnica
\titulacao{Bacharel}             %  TADS e CC = Bacharel; ETEC = Tecnólogo

\titulo{\UTFPRdocumentodata} % não alterar
\title{Intership Report}      % não alterar

\autor{Felipe Divensi Eckl} % autor do trabalho (obs: não usar colchetes nem neste nem em outros campos)
\cita{[DIVENSI, Felipe]} % sobrenome em maiusculas primeiro, seguindo de uma vírgula, seguido do nome do autor

\palavraschave{quadricóptero, multi-rotor, VTOL} % palavras-chave do trabalho
\keywords{quadrotor, multi-rotor, VTOL} % palavras-chave do trabalho em ingles


\comentario{\UTFPRdocumentodata\ apresentado à disciplina de Estágio Supervisionado, do Curso \UTFPRprogramadata\ da \ABNTinstituicaodata, como requisito parcial para obtenção do título de \UTFPRtitulacaodata.}

\orientador{Everton Coimbra de Araújo} % nome do orientador do trabalho

% \primeiroassina{Fabiano Da Rosa\\ Supervisor do Estágio - Frimesa Cooperativa Central}
\segundoassina{Prof. Everton Coimbra de Araújo\\ Orientador(a) do Estágio}
\terceiroassina{Felipe Divensi Eckl\\ Estagiário(a) }
\quartoassina{Prof. Neylor Michel\\ Responsável pela Atividade de Estágio do Curso}

% a seguir, insira o nome da empresa, o período e o total de horas do seu estágio
\textoaprovacao{
\UTFPRdocumentodata\ curricular supervisionado, realizado na Frimesa Cooperativa Central, no período de 15 de Setembro de 2017 a 12 de Março de 2018, perfazendo aproximadamente 643 horas.
}

\local{Medianeira} % cidade
\data{\the\year} % ano automatico

% desativa hifenizacao mantendo o texto justificado.
\tolerance=1
\emergencystretch=\maxdimen
\hyphenpenalty=10000
\hbadness=10000
\sloppy

\definecolor{laranjautfpr}{cmyk}{0.0, 0.2, 1.0, 0.0}
\usepackage{enumitem}
\setlist{leftmargin=2cm}
\setlist{nosep}


%---------- Inicio do Documento ----------
\begin{document}


\capa % geracao automatica da capa
\folhaderosto % geracao automatica da folha de rosto

\termodeentrega

% listas (opcionais, mas recomenda-se a partir de 5 elementos)
\listadefiguras % geracao automatica da lista de figuras
\listadetabelas % geracao automatica da lista de tabelas
%\listadequadros % adivinhe :)
\listadesiglas % geracao automatica da lista de siglas
%\listadesimbolos % geracao automatica da lista de simbolos

% sumario
\sumario % geracao automatica do sumario


%---------- Inicio do Texto ----------
% recomenda-se a escrita de cada capitulo em um arquivo texto separado (exemplo: intro.tex, fund.tex, exper.tex, concl.tex, etc.) e a posterior inclusao dos mesmos no mestre do documento utilizando o comando \input{}, da seguinte forma:

%\setcounter{page}{48}

\setlength{\parskip}{0.0cm}
\chapter{Introdução}\label{ch:intro}

A Frimesa Cooperativa Central atua no setor alimentício, que em 2017 completou 40 anos de mercado. De acordo com um relatório público de 2017, a Frimesa está entre as 300 maiores empresa do Brasil, tendo em seu quadro, aproximadamente 7 mil colaboradores diretos e faturando aproximadamente R\$ 2.83 bilhões. Possui como meta para 2018 3.2 bilhões de faturamento e com projeção estratégica de crescimento da empresa para R\$ 5.0 bilhões para 2022 \cite{relatorio2017}.

% Estamos terminando 2019 e você falou sobre 2018

A Unidade de Medianeira-PR é a sede da empresa, onde os departamentos administrativos responsáveis por toda a empresa estão localizados, como o departamento de Tecnologia da Informação \sigla{Ti}{Tecnologia da Informação}. O TI, de acordo com o relatório % QUE RELATÓRIO?, possui como objetivo ''Manter atualizada a Tecnologia da Informação da Frimesa.``.

O departamento de TI é subdividido em três setores: Suporte, Infraestrutura e Desenvolvimento de Sistemas. O último sendo o setor onde o estágio foi realizado.

\section{Atividades previstas}

As atividades previstas no plano de estágio incluem a criação e manutenção de relatorios, telas, criação de procedimentos e rotinas no Oracle E-Business Suite, utilizando PL/SQL. Programação nas linguagens Java, Shell Script, PHP e HTML, entre outras atividades a critério do superior imediato.

\section{Identificação}

Nome do acadêmico: Felipe Divensi Eckl.

Nome do orientador: Everton Coimbra de Araújo

Instituição de ensino: Universidade Tecnológica Federal do Paraná - UTFPR

Endereço da Instituição: 
Avenida Brasil, 4232 CEP 85884-000 - Caixa Postal 271.
Telefone Geral +55 (45) 3240-8000 - Fax : +55 (45) 3240-8101.
Medianeira - PR – Brasil.

Nome do curso: \UTFPRprogramadata.

Empresa: Frimesa Cooperativa Central.

Ramo de atividade da empresa: Análise e Desenvolvimento de Sistemas.

Endereço da empresa: 
Rua Bahia, 159 CEP 85884-000
Telefone +55 (45) 3264-8000 - Fax : +55 (45) 3264-8028
Medianeira - PR – Brasil

\section{Organização do documento}

Esse documento será organizado da seguinte forma. O Capítulo \ref{cap:metod} apresenta os materiais e métodos que foram utilizados no estágio. O Capítulo \ref{cap:result} irá falar sobre os projetos executados, como eles foram executados e quais os resultados obtidos. O Capítulo \ref{cap:concl} irá incluir algumas conclusões fazendo uma dissetação sobre a experiência no estágio.


\chapter{Metodologia} \label{cap:metod}

Neste capítulo serão mostrados os materiais, tecnologias e métodos de programação que foram utilizadas no estágio. 

\section{Hardware e Software}

O computador que foi utilizado no estágio foi um HP 260 G1 com um processador Intel Core i3-4030U que originalmente veio com 4GB de RAM, mas logo se percebeu que não era o suficiente para executar as ferramentas utilizadas com um bom desempenho e foi atualizado para 8GB de RAM.

O computador utilizado possuia o sistema operacional Microsoft Windows 10 Pro instalado. Este sistema está ligado ao \sigla{AD DS}{Active Directory Domain Services} (Active Directory Domain Services) da empresa. Active Directory é um serviço de diretório, ou seja, um sistema que gerencia recursos locais como impressoras, diretórios de usuário, e computadores, entre outros. Distribui nomes de domínio na rede para dispositivos e autentica e autoriza usuários na rede.

\section{Linguagens e Ambientes de Programação}

As principais linguagens de programação utilizadas para desenvolver projetos foram: PHP, JavaScript, Java, PL/SQL.

\sigla{PHP}{PHP Hypertext Preprocessor} (PHP Hypertext Preprocessor) é uma linguagem de programação baseada em um  processador de hipertexto Open Source de propósito geral, que é muito utilizada no  desenvolvimento web para gerar arquivos HTML\footnote{http://php.net/docs.php} . Em conjunto com o PHP foi utilizado o \textit{framework} de desenvolvimento web CakePHP versões 2 e 3. Este framework é baseado no padrão arquitetural \sigla{MVC}{Model-View-Controller} (Model-View-Controller)\footnote{https://book.cakephp.org/3.0/en/index.html}.

O arquitetura MVC separa a aplicação em modelos, que são os objetos que implementam a lógica de comunicação com uma fonte de dados, \textit{views}, que são a parte da aplicação responsável pela \sigla{UI}{User Interface} (User Interface, a interface do usuário), e controladoras que são responsáveis pela lógica da apliacação, manipulando os modelos e fazendo a manipulação das \textit{views}\footnote{https://msdn.microsoft.com/en-us/library/dd381412(v=vs.108).aspx}.

JavaScript é uma linguagem de alto-nível interpretada multi-paradigmas, que suporta programação funcional, imperativa e orientada a eventos. Implementações desta linguagem são encontradas na maioria dos navegadores de internet, no ambiente de desenvolvimento Node.js, em bancos de dados não relacionais como MongoDB e o Apache CouchDB, entre outros. Os padrões da linguagem JavaScript são definidos pelo ECMAScript que é desenvolvido pela organização de padrões ECMA International\footnote{https://developer.mozilla.org/bm/docs/Web/JavaScript}.

Java é uma linguagem Orientada a Objetos Open Source, atualmente mantida pela Oracle, criada com a mentalidade ''Write once, run anywhere``, utilizando o conceito de máquinas virtuais para atingir este objetivo. As aplicações desta linguagem são compliladas para o Java bytecode, que são o conjunto de instruções do Java Virtual Machine (\sigla{JVM}{Java Virtual Machine}). Implementações do JVM são feitas em vários Sistemas Operacionais e arquiteturas de processador, então, hipoteticamente, um programa criado em Java pode executar em qualquer dispositivo que tenha um JVM.

PL/SQL é uma linguagem procedural proprietária da Oracle. Programas feitos em PL/SQL são compilados pelo servidor do Oracle Database. A sintaxe do PL/SQL é baseada no SQL. \sigla{SQL}{Structured Query Language} (Structured Query Language) é uma linguagem usada em bancos de dados relacionais para acessar e manipular dados\footnote{http://www.oracle.com/technetwork/database/features/plsql/index.html}.

\section{Métodos}

A metodologia utilizada no setor de gestão da empresa é o \sigla{ITIL}{Information Technology Infrastructure Library} (Information Technology Infrastructure Library) que é um conjunto de melhores práticas para o alinhamento de serviços de TI  \cite{gallacher2012itil}. O ITIL possui recomendações para processos, procedimentos e tarefas na área, independente da organização em que este é aplicado, ou das tecnologias utilizadas. 

Para otimizar a aplicação do ITIL na empresa, é utilizada uma aplicação web desenvolvida internamente chamada Serviço de Gestão de Chamados (\sigla{SGC}{Serviço de Gestão de Chamados}). Nesta aplicação as tarefas a serem cumpridas são denominadas chamados, que são atribuídas aos analistas da empresa, junto com informações como prioridade, área de origem e outros.

% NÃO ENTENDI PORQUE DISSO AQUI
\section{Considerações Finais}

Neste capítulo foram mostrados os materiais, tecnologias e métodos de programação que foram utilizadas no estágio. No próximo capítulo será mostrado como estes foram utilizados, exibindo alguns projetos trabalhados e os resultados obtidos com eles.

\chapter{RESULTADOS E DISCUSSÕES} \label{cap:result}

Neste capítulo serão mostrados alguns dos projetos trabalhados e os resultados obtidos com o estágio.

\section{Projetos Trabalhados}

\subsection{Aplicação de Quiz Responsiva}

Um dos projetos desenvolvidos foi uma aplicação Web Responsiva para um Quiz da empresa (Figura \ref{fig:quiz-1}). 

% FALE SOBRE O APP. Para que serve? Quem usa? Faça isso para todos relacionados neste capítulo.


Após receber os requerimentos do projeto, foram decididas as tecnologias que seriam utilizadas na aplicação. Foi utilizado a linguagem de programação PHP com o framework CakePHP versão 3 em conjunto com o banco de dados MySQL. Também foi decidido que a página na qual os usuários que fossem interagir seria uma única página, e seu conteúdo seria atualizado utilizando JavaScript para se comunicar com uma WebAPI.

\begin{figure}[h!]
  \centering
  \includegraphics[width=0.7\textwidth]{Imagens/screenshots/quiz-1.png}
  \caption[Aplicação de Quiz Responsiva]{Aplicação de Quiz Responsiva}
  \label{fig:quiz-1}
  \fonte{(Autoria Própria)}
\end{figure}

\subsection{Portal da Intranet}

O projeto de criação do Portal da Intranet responsivo foi desenvolvido para substituir as funcionalidades de um portal anterior que fazia o mesmo papel. Este que é a página inicial da intranet da empresa com links a diversos sistemas, funcionalidades e manuais que são utilizados por funcionários de diversas áreas.

Também foi criada uma área administrativa para que os usuários que possuam permissões administrativas possam cadastrar os links e imagens que aparecem no portal. 

Este portal foi desenvolvido utilizando o \textit{framework} CakePHP versão 3, em conjunto com o banco de dados MySQL, também foi utilizado JavaScript para fazer animações e partes interativas das páginas, como um carrossel de imagens e o carregamento das notícias na página principal.

% apresente a figura


\begin{figure}[h!]
  \centering
  \includegraphics[width=0.7\textwidth]{Imagens/screenshots/portal-2.png}
  \caption[Portal da Intranet]{Portal da Intranet}
  \label{fig:portal-1}
  \fonte{(Autoria Própria)}
\end{figure}


\subsection{Portal do Representante Responsivo}

O projeto de criação do Portal de Representantes responsivo, Figura \ref{fig:portalrep-1}, foi criado para centralizar todas as informações e links para que os representantes da empresa possam utilizar enquanto estiverem trabalhando remotamente. Para isto, um dos principais requisitos da página é que ela fosse responsiva, e funcionasse em qualquer dispositivo Android e iOS atualizado. 

Esta página também teve uma página administrativa criada para que as informações deste fossem cadastradas. Este portal também foi desenvolvido utilizando o framework CakePHP versão 3 e o banco de dados MySQL, utilizando Javascript para interatividade na página inicial.

% está usando muito TAMBÉN


\begin{figure}[h!]
  \centering
  \includegraphics[width=0.7\textwidth]{Imagens/screenshots/portal-2.png}
  \caption[Portal do Representante]{Portal do Representante}
  \label{fig:portalrep-1}
  \fonte{(Autoria Própria)}
\end{figure}



\subsection{Manutenção Sistema de Chamados}

Parte das tarefas do estágio foi dar manutenção ao Sistema de Gestão de Chamados do setor de Tecnologia da Informação. Este sistema (também desenvolvido e mantido por outros funcionários) é responsável pela aplicação da metodologia ITIL da empresa. Neste foram desenvolvidas novas telas, atendendo a requisitos para melhoria do mesmo.

% E ONDE ESTÃO AS FIGURAS DO QUE FEZ? Precisa referenciar, indicar elas no texto. E também é preciso comentar sobre elas, o que são e representam.

\begin{figure}[h!]
  \centering
  \includegraphics[width=1\textwidth]{Imagens/screenshots/sgc-1.png}
  \caption[Sistema de Gestão de Chamados]{Sistema de Gestão de Chamados}
  \label{fig:sgc-1}
  \fonte{(Autoria Própria)}
\end{figure}


\chapter{Conclusões} \label{cap:concl}

Após a conclusão do estágio, é possível afirmar que foi cumprido o proposto no plano do mesmo. Este foi uma grande oportunidade para desenvolvimento pessoal, aonde foi possível pôr em prática o aprendido em diversas disciplinas do curso, como Bancos de Dados I e II, Tecnologia em Desenvolvimento de Sistemas, Engenharia de Requisitos e Programação Orientada a Objetos.

\section{Trabalhos Futuros/Continuação do Trabalho}

Os trabalhos desenvolvidos foram utilizados em produção até a data de término do estágio, portanto como trabalhos futuros cabem a manutenção e desenvolvimento de novas funcionalidades conforme requerido.


%---------- Referencias ----------
\clearpage % this is need for add +1 to pageref of bibstart used in 'ficha catalografica'.
\label{bibstart}
\bibliography{bibliografia} % geracao automatica das referencias a partir do arquivo bibliografia.bib
\label{bibend}



% --------- Ordenacao Afabetica da Lista de siglas --------
%\textbf{* Observa\c{c}\~oes:} a ordenacao alfabetica da lista de siglas ainda nao eh realizada de forma automatica, porem
% eh possivel se de realizar isto manualmente. Duas formas:
%
% ** Primeira forma)
%    A ordenacao eh feita com o auxilio do comando 'sort', disponivel em qualquer
% sistema Linux e UNIX, e tambem em sistemas Windows se instalado o coreutils (http://gnuwin32.sourceforge.net/packages/coreutils.htm)
% comandos para compilar e ordenar, supondo que seu arquivo se chame 'dissertacao.tex':
%
%      $ latex dissertacao
%      $ bibtex dissertacao && latex dissertacao
%      $ latex dissertacao
%      $ sort dissertacao.lsg > dissertacao.lsg.tmp
%      $ mv dissertacao.lsg.tmp dissertacao.lsg
%      $ latex dissertacao
%      $ dvipdf dissertacao.dvi
%
%
% ** Segunda forma)
%\textbf{Sugest\~ao:} crie outro arquivo .tex para siglas e utilize o comando \sigla{sigla}{descri\c{c}\~ao}.
%Para incluir este arquivo no final do arquivo, utilize o comando \input{arquivo.tex}.
%Assim, Todas as siglas serao geradas na ultima pagina. Entao, devera excluir a ultima pagina da versao final do arquivo
% PDF do seu documento.


%-------- Citacoes ---------
% - Utilize o comando \citeonline{...} para citacoes com o seguinte formato: Autor et al. (2011).
% Este tipo de formato eh utilizado no comeco do paragrafo. P.ex.: \citeonline{autor2011}

% - Utilize o comando \cite{...} para citacoeses no meio ou final do paragrafo. P.ex.: \cite{autor2011}



%-------- Titulos com nomes cientificos (titulo, capitulos e secoes) ----------
% Regra para escrita de nomes cientificos:
% Os nomes devem ser escritos em italico, 
%a primeira letra do primeiro nome deve ser em maiusculo e o restante em minusculo (inclusive a primeira letra do segundo nome).
% VEJA os exemplos abaixo.
% 
% 1) voce nao quer que a secao fique com uppercase (caixa alta) automaticamente:
%\section[nouppercase]{\MakeUppercase{Estudo dos efeitos da radiacao ultravioleta C e TFD em celulas de} {\textit{Saccharomyces boulardii}}
%
% 2) por padrao os cases (maiusculas/minuscula) sao ajustados automaticamente, voce nao precisa usar makeuppercase e afins.
% \section{Introducao} % a introducao sera posta no texto como INTRODUCAO, automaticamente, como a norma indica.


\end{document}
