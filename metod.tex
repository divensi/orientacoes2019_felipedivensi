\chapter{Metodologia} \label{cap:metod}

Neste capítulo serão mostrados os materiais, tecnologias e métodos de programação que foram utilizadas no estágio. 

\section{Hardware e Software}

O computador que foi utilizado no estágio foi um HP 260 G1 com um processador Intel Core i3-4030U que originalmente veio com 4GB de RAM, mas logo se percebeu que não era o suficiente para executar as ferramentas utilizadas com um bom desempenho e foi atualizado para 8GB de RAM.

O computador utilizado possuia o sistema operacional Microsoft Windows 10 Pro instalado. Este sistema está ligado ao \sigla{AD DS}{Active Directory Domain Services} (Active Directory Domain Services) da empresa. Active Directory é um serviço de diretório, ou seja, um sistema que gerencia recursos locais como impressoras, diretórios de usuário, e computadores, entre outros. Distribui nomes de domínio na rede para dispositivos e autentica e autoriza usuários na rede.

\section{Linguagens e Ambientes de Programação}

As principais linguagens de programação utilizadas para desenvolver projetos foram: PHP, JavaScript, Java, PL/SQL.

\sigla{PHP}{PHP Hypertext Preprocessor} (PHP Hypertext Preprocessor) é uma linguagem de programação baseada em um  processador de hipertexto Open Source de propósito geral, que é muito utilizada no  desenvolvimento web para gerar arquivos HTML\footnote{http://php.net/docs.php} . Em conjunto com o PHP foi utilizado o \textit{framework} de desenvolvimento web CakePHP versões 2 e 3. Este framework é baseado no padrão arquitetural \sigla{MVC}{Model-View-Controller} (Model-View-Controller)\footnote{https://book.cakephp.org/3.0/en/index.html}.

O arquitetura MVC separa a aplicação em modelos, que são os objetos que implementam a lógica de comunicação com uma fonte de dados, \textit{views}, que são a parte da aplicação responsável pela \sigla{UI}{User Interface} (User Interface, a interface do usuário), e controladoras que são responsáveis pela lógica da apliacação, manipulando os modelos e fazendo a manipulação das \textit{views}\footnote{https://msdn.microsoft.com/en-us/library/dd381412(v=vs.108).aspx}.

JavaScript é uma linguagem de alto-nível interpretada multi-paradigmas, que suporta programação funcional, imperativa e orientada a eventos. Implementações desta linguagem são encontradas na maioria dos navegadores de internet, no ambiente de desenvolvimento Node.js, em bancos de dados não relacionais como MongoDB e o Apache CouchDB, entre outros. Os padrões da linguagem JavaScript são definidos pelo ECMAScript que é desenvolvido pela organização de padrões ECMA International\footnote{https://developer.mozilla.org/bm/docs/Web/JavaScript}.

Java é uma linguagem Orientada a Objetos Open Source, atualmente mantida pela Oracle, criada com a mentalidade ''Write once, run anywhere``, utilizando o conceito de máquinas virtuais para atingir este objetivo. As aplicações desta linguagem são compliladas para o Java bytecode, que são o conjunto de instruções do Java Virtual Machine (\sigla{JVM}{Java Virtual Machine}). Implementações do JVM são feitas em vários Sistemas Operacionais e arquiteturas de processador, então, hipoteticamente, um programa criado em Java pode executar em qualquer dispositivo que tenha um JVM.

PL/SQL é uma linguagem procedural proprietária da Oracle. Programas feitos em PL/SQL são compilados pelo servidor do Oracle Database. A sintaxe do PL/SQL é baseada no SQL. \sigla{SQL}{Structured Query Language} (Structured Query Language) é uma linguagem usada em bancos de dados relacionais para acessar e manipular dados\footnote{http://www.oracle.com/technetwork/database/features/plsql/index.html}.

\section{Métodos}

A metodologia utilizada no setor de gestão da empresa é o \sigla{ITIL}{Information Technology Infrastructure Library} (Information Technology Infrastructure Library) que é um conjunto de melhores práticas para o alinhamento de serviços de TI  \cite{gallacher2012itil}. O ITIL possui recomendações para processos, procedimentos e tarefas na área, independente da organização em que este é aplicado, ou das tecnologias utilizadas. 

Para otimizar a aplicação do ITIL na empresa, é utilizada uma aplicação web desenvolvida internamente chamada Serviço de Gestão de Chamados (\sigla{SGC}{Serviço de Gestão de Chamados}). Nesta aplicação as tarefas a serem cumpridas são denominadas chamados, que são atribuídas aos analistas da empresa, junto com informações como prioridade, área de origem e outros.

% NÃO ENTENDI PORQUE DISSO AQUI
\section{Considerações Finais}

Neste capítulo foram mostrados os materiais, tecnologias e métodos de programação que foram utilizadas no estágio. No próximo capítulo será mostrado como estes foram utilizados, exibindo alguns projetos trabalhados e os resultados obtidos com eles.
